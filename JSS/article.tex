\documentclass[article]{jss}

%%%%%%%%%%%%%%%%%%%%%%%%%%%%%%
%% declarations for jss.cls %%%%%%%%%%%%%%%%%%%%%%%%%%%%%%%%%%%%%%%%%%
%%%%%%%%%%%%%%%%%%%%%%%%%%%%%%

%% almost as usual
\author{Christina Knudson\\ University of St. Thomas \And 
        Charles Geyer\\University of Minnesota \And 
        Galin Jones\\University of Minnesota}
\title{ Likelihood-Based Inference for Generalized Linear Mixed Models: Inference with Package \pkg{glmm}}

%% for pretty printing and a nice hypersummary also set:
\Plainauthor{Christina Knudson, Charles Geyer, Galin Jones} %% comma-separated
\Plaintitle{A Capitalized Title: Something about a Package foo} %% without formatting
\Shorttitle{\pkg{glmm}: Likelihood-Based Inference for Generalized Linear Mixed Models} %% a short title (if necessary)

%% an abstract and keywords
\Abstract{
The R package \pkg{glmm} enables all  likelihood-based inference  for generalized linear mixed models with a canonical link by approximating the entire likelihood function and two derivatives. The model fitting function of \pkg{glmm} calculates and maximizes the Monte Carlo likelihood approximation to find Monte Carlo maximum likelihood estimates for the fixed effects and variance components. Additionally, the value, gradient vector, and Hessian matrix of the MCLA are calculated at the Monte Carlo maximum likelihood estimates. \pkg{glmm} also produces the variance-covariance matrix of the estimates, confidence intervals for the parameters, and Monte Carlo standard errors of the Monte Carlo maximum likelihood estimates.
}
\Keywords{generalized linear mixed models, maximum likelihood, likelihood-based inference, mixed models, likelihood approximation, Monte Carlo}
\Plainkeywords{generalized linear mixed models, maximum likelihood, likelihood-based inference, mixed models, likelihood approximation, Monte Carlo}
 %% without formatting
%% at least one keyword must be supplied

%% publication information
%% NOTE: Typically, this can be left commented and will be filled out by the technical editor
%% \Volume{50}
%% \Issue{9}
%% \Month{June}
%% \Year{2012}
%% \Submitdate{2012-06-04}
%% \Acceptdate{2012-06-04}

%% The address of (at least) one author should be given
%% in the following format:
\Address{
  Christina Knudson\\
  Department of Mathematics\\
  Faculty of  Statistics\\
 University of St. Thomas\\
2115 Summit Avenue, Saint Paul, Minnesota\\
  E-mail: \email{christina@umn.edu}\\
  URL: \url{http://cknudson.com/}
}
%\Address{
% Galin Jones\\
%School of Statistics\\
%  Faculty of Statistics\\
% University of Minnesota\\
%224 Church Street Southeast, Minneapolis, Minnesota \\
%  E-mail: \email{galin@umn.edu}\\
%  URL: \url{http://users.stat.umn.edu/~galin/}
%}
%% It is also possible to add a telephone and fax number
%% before the e-mail in the following format:
%% Telephone: +43/512/507-7103
%% Fax: +43/512/507-2851

%% for those who use Sweave please include the following line (with % symbols):
%% need no \usepackage{Sweave.sty}

%% end of declarations %%%%%%%%%%%%%%%%%%%%%%%%%%%%%%%%%%%%%%%%%%%%%%%


\begin{document}

%% include your article here, just as usual
%% Note that you should use the \pkg{}, \proglang{} and \code{} commands.

\section{Introduction}
The generalized linear mixed model (GLMM) is an extension of both the generalized linear model and the linear mixed model; the model incorporates  fixed and random effects as well as a response from an exponential family.   GLMMs were first discussed by \citet{stiratelli:laird:ware:1984} and are now  used in a variety of disciplines. Despite their widespread use, traditional methods of frequentist likelihood-based inference are not generally available. The challenge lies in the likelihood function for GLMMs: because the likelihood cannot depend on the random effects, the likelihood is an  integral that is often intractable (details    in FIX). Due to this challenge, most methodology and software for  GLMMs perform little more than maximum likelihood (such as Monte Carlo EM)  or do not perform likelihood-based inference at all (such as penalized quasi-likelihood). 
Our goal is to enable all types of frequentist likelihood-based inference. This includes (but is not limited to) calculating Fisher information, performing hypothesis tests, and constructing confidence intervals.  To perform all types of frequentist likelihood-based inference, the entire likelihood function is necessary.\\


With this goal in mind, we have created an R package \pkg{glmm} \citep{glmm:package} that approximates the entire likelihood function for a GLMM with either a Bernoulli or Poisson response. This package uses  the method of
 Monte Carlo likelihood approximation (MCLA) \citep{geyer:1994, geyer:thom:1992,  sung:geyer:2007}. 
%This method relies on the  choice of an importance sampling distribution,  a distribution from which simulated random effects are drawn. These generated random effects are then used to approximate the entire likelihood function. 
Because MCLA approximates the entire likelihood, \pkg{glmm}  enables  every type of likelihood-based inference. \\

%For more details on MCLA, see \ref{sec:mclageneral}.

Before the release of R package \pkg{glmm}, no publicly-available software  implementing MCLA  produced accurate maximum likelihood estimates for models of non-trivial complexity. For example, the \texttt{bernor}  package \citep{sung:geyer:2007} implements MCLA but cannot accurately perform maximum likelihood for the benchmark data set from a salamander mating experiment (described in section \ref{sec:examples}). 
%The  absence of MCLA software highlights a   challenge from which the procedure suffers:  finding an importance sampling distribution that works well in practice.  
%We propose  an importance sampling distribution to be used in implementing MCLA for GLMMs (Equation {eq:ftwiddle});  
We present   \pkg{glmm}  and demonstrate in section \ref{sec:howto} how to use the package to perform maximum likelihood, test hypotheses, calculate confidence intervals, and calculate Monte Carlo standard errors for the estimates.  


\subsection{Examples}\label{sec:examples}

\subsubsection{The salamander data}

Researchers at the University of Chicago conducted an experiment on a single species of salamanders  in 1986.
 \citet[Section 14.5]{mcc:nelder:1989} presented the experiment and data, and \citet{karim:zeger:1992} proposed a model they call ``Model A.'' The salamander data and model have become a benchmark in the GLMM universe;  the data have been modeled by many researchers including \citet{booth:hobert:1999},  \citet{bres:clay:1993},   \citet{karim:zeger:1992}, \citet{mcc:nelder:1989},  \citet{schall:1991},  \citet{sung:geyer:2007}, and \citet{wolfinger:oconnell:1993}.\\




Before the experiment began, female salamanders and male salamanders of the same species were collected from two locations. The salamanders were categorized into populations named ``Rough Butt'' and ``White Side'' based on their location of origin. The scientific goal was to determine whether salamanders were more likely to mate with those from their own population or whether they were just as likely to mate with salamanders from either population. More specifically, scientsts sought to compare the odds of mating for each type of  cross: female Rough Butt salamanders and male Rough Butt salamanders (denoted RR), female Rough Butt salamanders and male White Side salamanders (denoted RW), female White Side salamanders and male Rough Butt salamanders (denoted WR), and female White Side salamanders and male White Side salamanders (denoted WW).    \\

Scientists ran an experiment three times. Each experiment consisted of trials conducted on two closed groups of 20 salamanders. That is, each experiment was conducted on 40 salamanders that were split into two closed groups.  Each group contained 5 female Rough Butts, 5 female White Sides, 5 male Rough Butts, and 5 male White Sides.  Each trial consisted of placing a  female salamander and a  male salamander in an isolated space together, then observing the binary response of interest: whether the salamanders mated.  Each female salamander participated in six trials with male salamanders from her closed group:  three trials with male White Side salamanders and three trials with male Rough Butt salamanders. Scientists paired salamanders from the same group; inter-group trials were not conducted. Thus, 60 trials were conducted on each closed group, each experiment consisted of 120 trials, and the overall dataset contains binary responses from 360 trials. \\


 To model these data with a GLMM,  ``Model A'' \citep{karim:zeger:1992} proposes
a random effect for each female salamander; a random effect for each male salamander; a fixed effect predictor for each of the four types of cross; and a Bernoulli response of whether the pair of salamanders mated, dependent on the type of cross, the female random effect, and the male random effect. 
 Each salamander's random effect is assumed to be independent of the others.  The male salamanders' random effects share one variance component  while the female salamanders' random effects share another variance component. These two variance components are later referred to as $\nu_M$ and $\nu_F$. The males' random effects are crossed with the females' random effects. Though the same salamanders were used in the first two experiments, the data are traditionally modeled assuming that new salamanders were used in each experiment \citep{booth:hobert:1999,  karim:zeger:1992,  mcc:nelder:1989}.\\

\subsubsection{The radish flower data}

The data in this example are a subset of the data collected by \citet{ridley:ellstrand:2010}. The scientists were interested in whether non-native radishes had adapted to the climate they had been growing in for the last 150 years. In other words, they wanted to compare two types of radish to see whether each type would grow just as well in their own climate as they would in the other climate. 

In this dataset, the response is the number of radish flowers. This is assumed to have a Poisson distribution.  \texttt{Site} is a categorical variable with two categories representing the two sites where plants were grown. The variable \texttt{Region} is categorical with two categories representing the two places in California from which the plants were taken. The variable \texttt{Pop} is a categorical variable representing the population of radish, and \texttt{Pop} is nested in \texttt{Region}. The variable \texttt{Block} is a categorical blocking variable nested in \texttt{Site}.  Following the example of \citet{ridley:ellstrand:2010}, \texttt{Block} and \texttt{Pop} are random while \texttt{Site} and \texttt{Region} are fixed. The scientists were interested in  the interaction between \texttt{Site} and \texttt{Region}, since that would indicate that radishes grow better in the area they have been grown during recent history.




\section{Monte Carlo Likelihood Approximation}

\subsection[Implementation of MCLA in glmm]{Implementation of MCLA in  \pkg{glmm}}

\section[How to use glmm]{How to use \pkg{glmm}}\label{sec:howto}


\subsection{Formatting the data}
\subsection{Using the model-fitting function}
\subsection{Using the summary output}
\subsection{Performing additional inference}


\section{Conclusion}

\bibliography{brref}



\end{document}
